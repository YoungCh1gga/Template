\documentclass[a4paper,11pt]{ctexbook} 
\usepackage[a4paper,scale=0.8,bindingoffset=8mm]{geometry} % A4纸大小,缩放80%,设置奇数页右边留空多一点
\usepackage{hyperref}      % 超链接
\usepackage{listings}      % 代码块
\usepackage{fancyhdr}      % 页眉页脚相关宏包
\usepackage{lastpage}      % 引用最后一页
\usepackage{amsmath,amsthm,amsfonts,amssymb,bm} %数学
\usepackage{graphicx}      % 图片
\usepackage{subcaption}    % 图片描述
\usepackage{longtable,booktabs} % 表格
\usepackage{xeCJK}

\usepackage{listings}
\usepackage{ctex}
\usepackage{color,xcolor}

% 用来设置附录中代码的样式



\lstset{
    basicstyle          =   \sffamily,          % 基本代码风格
    keywordstyle        =   \bfseries,          % 关键字风格
    commentstyle        =   \rmfamily\itshape,  % 注释的风格,斜体
    stringstyle         =   \ttfamily,  % 字符串风格
    flexiblecolumns,                % 别问为什么,加上这个
    numbers             =   left,   % 行号的位置在左边
    showspaces          =   false,  % 是否显示空格,显示了有点乱,所以不现实了
    numberstyle         =   \zihao{-5}\ttfamily,    % 行号的样式,小五号,tt等宽字体
    showstringspaces    =   false,
    captionpos          =   t,      % 这段代码的名字所呈现的位置,t指的是top上面
    frame               =   l,   % 显示边框
}

\lstdefinestyle{Python}{
    language        =   Python, % 语言选Python
    basicstyle      =   \zihao{-5}\ttfamily,
    numberstyle     =   \zihao{-5}\ttfamily,
    keywordstyle    =   \color{blue},
    keywordstyle    =   [2] \color{teal},
    stringstyle     =   \color{magenta},
    commentstyle    =   \color{red}\ttfamily,
    breaklines      =   true,   % 自动换行,建议不要写太长的行
    columns         =   fixed,  % 如果不加这一句,字间距就不固定,很丑,必须加
    basewidth       =   0.5em,
}
\lstdefinestyle{cpp}{
    language        =   C++, % 语言选C++
    basicstyle      =   \zihao{-5}\ttfamily,
    numberstyle     =   \zihao{-5}\ttfamily,
    keywordstyle    =   \color{blue},
    keywordstyle    =   [2] \color{teal},
    stringstyle     =   \color{magenta},
    commentstyle    =   \color{red}\ttfamily,
    breaklines      =   true,   % 自动换行,建议不要写太长的行
    columns         =   fixed,  % 如果不加这一句,字间距就不固定,很丑,必须加
    basewidth       =   0.5em,
}

%\lstinputlisting[
%     style       =   Python,
%     caption     =   {\bf ff.py},
%     label       =   {ff.py}
% ]{../src/duke/ff.py}


\pagestyle{fancy}         % 页眉页脚风格
\fancyhf{}                % 清空当前设置
\fancyfoot[C]{\thepage\ / \pageref{LastPage}}%页脚中间显示 当前页 / 总页数,把\label{LastPage}放在最后
\fancyhead[RO,LE]{\thepage}% 页眉奇数页左边,偶数页右边显示当前页
\begin{document} 
  \begin{titlepage}       % 封面
    \centering
    \vspace*{\baselineskip}
    \rule{\textwidth}{1.6pt}\vspace*{-\baselineskip}\vspace*{2pt}
    \rule{\textwidth}{0.4pt}\\[\baselineskip]
    {\LARGE Algos @CUGB 2020\\[\baselineskip]\small for CUGBACM}
    \\[0.2\baselineskip]
    \rule{\textwidth}{0.4pt}\vspace*{-\baselineskip}\vspace{3.2pt}
    \rule{\textwidth}{1.6pt}\\[\baselineskip]
    \scshape

    \begin{figure}[!htb]
        \centering
        %\includegraphics[width=0.3\textwidth]{icpc}    % 当前tex文件同一目录下名为icpc的任意格式图片
    \end{figure}

    % \vspace*{3\baselineskip}
    % Edited by \\[\baselineskip] {Byjiang\par}
    % {Team \Large Byjiang \normalsize{at CUGB}\par }
    \vfill
    {\scshape 2020} \\{\large CUGBACM}\par
  \end{titlepage}
 \newpage            % 封面背后空白页
\tableofcontents     % 目录

\setcounter{page}{1}

%\input{math/math}
\section{DataStruct}
\subsection{Chtholly.cpp}
\lstinputlisting[style=cpp]{./DataStruct/Chtholly.cpp}
\subsection{DSU.cpp}
\lstinputlisting[style=cpp]{./DataStruct/DSU.cpp}
\subsection{LazySegmentTree.cpp}
\lstinputlisting[style=cpp]{./DataStruct/LazySegmentTree.cpp}
\subsection{Mo.cpp}
\lstinputlisting[style=cpp]{./DataStruct/Mo.cpp}
\subsection{NearestPointPair.cpp}
\lstinputlisting[style=cpp]{./DataStruct/NearestPointPair.cpp}
\subsection{PointDivideAndConquer1.cpp}
\lstinputlisting[style=cpp]{./DataStruct/PointDivideAndConquer1.cpp}
\subsection{PointDivideAndConquer2.cpp}
\lstinputlisting[style=cpp]{./DataStruct/PointDivideAndConquer2.cpp}
\subsection{Segtree.cpp}
\lstinputlisting[style=cpp]{./DataStruct/Segtree.cpp}
\subsection{SegtreeNoneRecursive.cpp}
\lstinputlisting[style=cpp]{./DataStruct/SegtreeNoneRecursive.cpp}
\subsection{SparseTable.cpp}
\lstinputlisting[style=cpp]{./DataStruct/SparseTable.cpp}
\subsection{TheKthFarPointPair.cpp}
\lstinputlisting[style=cpp]{./DataStruct/TheKthFarPointPair.cpp}
\subsection{Trie01.cpp}
\lstinputlisting[style=cpp]{./DataStruct/Trie01.cpp}
\subsection{dsu\_on\_tree.cpp}
\lstinputlisting[style=cpp]{./DataStruct/dsu_on_tree.cpp}
\subsection{fenwick.cpp}
\lstinputlisting[style=cpp]{./DataStruct/fenwick.cpp}
\subsection{fhq-Treap(区间).cpp}
\lstinputlisting[style=cpp]{./DataStruct/fhq-Treap(区间).cpp}
\subsection{fhq-Treap.cpp}
\lstinputlisting[style=cpp]{./DataStruct/fhq-Treap.cpp}
\subsection{jls线段树.cpp}
\lstinputlisting[style=cpp]{./DataStruct/jls线段树.cpp}
\subsection{segment\_tree3.cpp}
\lstinputlisting[style=cpp]{./DataStruct/segment_tree3.cpp}
\subsection{主席树.cpp}
\lstinputlisting[style=cpp]{./DataStruct/主席树.cpp}
\subsection{区间覆盖.cpp}
\lstinputlisting[style=cpp]{./DataStruct/区间覆盖.cpp}
\subsection{带权并查集.cpp}
\lstinputlisting[style=cpp]{./DataStruct/带权并查集.cpp}
\subsection{替罪羊.cpp}
\lstinputlisting[style=cpp]{./DataStruct/替罪羊.cpp}
\subsection{树剖.cpp}
\lstinputlisting[style=cpp]{./DataStruct/树剖.cpp}
\subsection{笛卡尔树.cpp}
\lstinputlisting[style=cpp]{./DataStruct/笛卡尔树.cpp}
\subsection{轻重链剖分.cpp}
\lstinputlisting[style=cpp]{./DataStruct/轻重链剖分.cpp}
\section{Geometry}
\subsection{Circle.cpp}
\lstinputlisting[style=cpp]{./Geometry/Circle.cpp}
\subsection{HalfPlane.cpp}
\lstinputlisting[style=cpp]{./Geometry/HalfPlane.cpp}
\subsection{Line.cpp}
\lstinputlisting[style=cpp]{./Geometry/Line.cpp}
\subsection{Point.cpp}
\lstinputlisting[style=cpp]{./Geometry/Point.cpp}
\subsection{PolygonAndConvex.cpp}
\lstinputlisting[style=cpp]{./Geometry/PolygonAndConvex.cpp}
\subsection{Triangle.cpp}
\lstinputlisting[style=cpp]{./Geometry/Triangle.cpp}
\subsection{mygeo.cpp}
\lstinputlisting[style=cpp]{./Geometry/mygeo.cpp}
\section{Graph}
\subsection{2sat.cpp}
\lstinputlisting[style=cpp]{./Graph/2sat.cpp}
\subsection{Graph.cpp}
\lstinputlisting[style=cpp]{./Graph/Graph.cpp}
\subsection{MaxAssignment.cpp}
\lstinputlisting[style=cpp]{./Graph/MaxAssignment.cpp}
\subsection{Mincost.cpp}
\lstinputlisting[style=cpp]{./Graph/Mincost.cpp}
\subsection{Tree.cpp}
\lstinputlisting[style=cpp]{./Graph/Tree.cpp}
\subsection{dijkstra.cpp}
\lstinputlisting[style=cpp]{./Graph/dijkstra.cpp}
\subsection{dinic.cpp}
\lstinputlisting[style=cpp]{./Graph/dinic.cpp}
\subsection{spfa.cpp}
\lstinputlisting[style=cpp]{./Graph/spfa.cpp}
\subsection{匈牙利.cpp}
\lstinputlisting[style=cpp]{./Graph/匈牙利.cpp}
\section{Math}
\subsection{China.cpp}
\lstinputlisting[style=cpp]{./Math/China.cpp}
\subsection{Euler.cpp}
\lstinputlisting[style=cpp]{./Math/Euler.cpp}
\subsection{FFT.cpp}
\lstinputlisting[style=cpp]{./Math/FFT.cpp}
\subsection{Lagrange.cpp}
\lstinputlisting[style=cpp]{./Math/Lagrange.cpp}
\subsection{Lucas.cpp}
\lstinputlisting[style=cpp]{./Math/Lucas.cpp}
\subsection{Miller-Rabin.cpp}
\lstinputlisting[style=cpp]{./Math/Miller-Rabin.cpp}
\subsection{NTT.cpp}
\lstinputlisting[style=cpp]{./Math/NTT.cpp}
\subsection{basic.cpp}
\lstinputlisting[style=cpp]{./Math/basic.cpp}
\subsection{binom.cpp}
\lstinputlisting[style=cpp]{./Math/binom.cpp}
\subsection{exgcd.cpp}
\lstinputlisting[style=cpp]{./Math/exgcd.cpp}
\subsection{xor\_basis.cpp}
\lstinputlisting[style=cpp]{./Math/xor_basis.cpp}
\subsection{公式.md}
\lstinputlisting[style=cpp]{./Math/公式.md}
\subsection{区间线性基.cpp}
\lstinputlisting[style=cpp]{./Math/区间线性基.cpp}
\subsection{取模gauss.cpp}
\lstinputlisting[style=cpp]{./Math/取模gauss.cpp}
\subsection{容斥.cpp}
\lstinputlisting[style=cpp]{./Math/容斥.cpp}
\subsection{异或gauss.cpp}
\lstinputlisting[style=cpp]{./Math/异或gauss.cpp}
\subsection{斐波那契.cpp}
\lstinputlisting[style=cpp]{./Math/斐波那契.cpp}
\subsection{求逆元.cpp}
\lstinputlisting[style=cpp]{./Math/求逆元.cpp}
\subsection{浮点型gauss.cpp}
\lstinputlisting[style=cpp]{./Math/浮点型gauss.cpp}
\subsection{第二类斯特林数.cpp}
\lstinputlisting[style=cpp]{./Math/第二类斯特林数.cpp}
\subsection{线性基类.cpp}
\lstinputlisting[style=cpp]{./Math/线性基类.cpp}
\subsection{除法分块.cpp}
\lstinputlisting[style=cpp]{./Math/除法分块.cpp}
\section{Others}
\subsection{BigNum2.cpp}
\lstinputlisting[style=cpp]{./Others/BigNum2.cpp}
\subsection{Simulated\_annealing.cpp}
\lstinputlisting[style=cpp]{./Others/Simulated_annealing.cpp}
\subsection{Z.cpp}
\lstinputlisting[style=cpp]{./Others/Z.cpp}
\subsection{bignum.cpp}
\lstinputlisting[style=cpp]{./Others/bignum.cpp}
\subsection{gen.py}
\lstinputlisting[style=cpp]{./Others/gen.py}
\subsection{makestd.cpp}
\lstinputlisting[style=cpp]{./Others/makestd.cpp}
\subsection{pai.py}
\lstinputlisting[style=cpp]{./Others/pai.py}
\subsection{sg函数.cpp}
\lstinputlisting[style=cpp]{./Others/sg函数.cpp}
\subsection{博弈.cpp}
\lstinputlisting[style=cpp]{./Others/博弈.cpp}
\subsection{威佐夫博弈.cpp}
\lstinputlisting[style=cpp]{./Others/威佐夫博弈.cpp}
\subsection{杜教BM.cpp}
\lstinputlisting[style=cpp]{./Others/杜教BM.cpp}
\subsection{欧拉函数.cpp}
\lstinputlisting[style=cpp]{./Others/欧拉函数.cpp}
\section{String}
\subsection{AhoCorasick.cpp}
\lstinputlisting[style=cpp]{./String/AhoCorasick.cpp}
\subsection{exkmp.cpp}
\lstinputlisting[style=cpp]{./String/exkmp.cpp}
\subsection{kmp.cpp}
\lstinputlisting[style=cpp]{./String/kmp.cpp}
\subsection{manacher.cpp}
\lstinputlisting[style=cpp]{./String/manacher.cpp}
\subsection{后缀数组.cpp}
\lstinputlisting[style=cpp]{./String/后缀数组.cpp}
\section{dp}
\subsection{数位dp.cpp}
\lstinputlisting[style=cpp]{./dp/数位dp.cpp}
\subsection{最长上升子序列.cpp}
\lstinputlisting[style=cpp]{./dp/最长上升子序列.cpp}


\end{document}
\label{LastPage}